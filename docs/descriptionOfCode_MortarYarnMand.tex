\subsection{Description of code: mortar contact for yarn-mandrel interactions}
Odin multibody dynamics research code \cite{odin2022} is written in C++ programming language. The philosophy of Odin relies on the use of nonlinear finite elements to represent the components of a multibody system. This is further accompanied by kinematic or contact constraints. The equations of motion are automatically assembled in Odin based on the specified choice of the mechanical analysis and further solved using the specified numerical scheme.\\

\subsubsection{Python code file} 
\begin{enumerate}
    \item The necessary modules are imported and Odin is initialized using \pythoninline{initialize_odin()}.
    \item A physical system is created which is the master container for all the components of the system, using \pythoninline{ps = create_physical_system()}. Additionally, the mechanical analysis and nodal frames are defined in the physical system. As a nodal frame is defined by translations and rotations, a \pythoninline{Vec3} (vector of 3 components) can be used for defining translations along with quaternions for rotations.
    \item The \pythoninline{JSON} file which configures the methods to be used (nonlinear solvers etc.) is loaded using \pythoninline{json_file = json.load(open())}.
    \item For defining the circular motion as the boundary condition for yarns, the radius \pythoninline{r_trajectory} is defined along with defining the radius of yarns \pythoninline{r}, length \pythoninline{l_beam} and mandrel dimensions. The number of finite elements to be used are defined as \pythoninline{nele_AB} and \pythoninline{nele_BC} for the yarns and mandrel beam respectively.
    \item The element and node labels are initialized as \pythoninline{elabel = 0} and \pythoninline{nlabel = 0}.
    \item For \pythoninline{nlabel = 0}, a grounded spherical joint has been defined by simply specifying the position of the joint and using the C++ element \pythoninline{Ground_SphericalJ} as :
    \pythonstyle
    \begin{tcolorbox}\begin{lstlisting}
    prop_spherical = SphericalJProps()
    prop_spherical.position = np.array(start_point_AB)
    ps.create_element("Ground_SphericalJ", elabel, [nlabel],
                        prop_spherical)
    elabel += 1 
    \end{lstlisting}\end{tcolorbox}
        The elabel has now been incremented using \pythoninline{elabel += 1}. In this way, no massless node is defined.
    \item The material properties and sectional stiffness coefficients of the beam are defined using the variable \pythoninline{params_beam}. 
    \item Using \pythoninline{beamAB_props}, the choice of using a tangent operator and geometric stiffness is defined with \pythoninline{True} or \pythoninline{False} as \pythoninline{beamAB_props.with_tg_operator = False}.
    \item The normal and binormal vectors of the orthonormal triad are defined as \pythoninline{n_0 = np.array([0, 1, 0])} ($Y$-axis) and \pythoninline{b_0 = np.array([0, 0, 1])} ($Z$-axis). Finally the beam is created in the physical system using:
    \pythonstyle
    \begin{tcolorbox}\begin{lstlisting}
    prepro.straight_beam(start_point_AB, end_point_AB, nele_AB, 
    elabel, nlabel, n_0, b_0, beamAB_props, ps)
    elabel += nele_AB 
    nlabel += nele_AB 
    n_beamAB_1 = nlabel
    \end{lstlisting}\end{tcolorbox}
        The \pythoninline{elabel} and \pythoninline{nlabel} are further incremented using \pythoninline{nele_AB} and \pythoninline{n_beamAB_1} is specified to label the last node of beam AB (yarn 1).
    \item Essential boundary conditions for circular motion are now defined using piecewise linear functions \pythoninline{PWLFunction()}. The prescribed circular motion is in the $YZ$ plane and therefore the remaining $4$ DoF are clamped. Also, the motion needs to be prescribed at position, velocity and acceleration levels as:
    \pythonstyle
    \begin{tcolorbox}\begin{lstlisting}
    bcx = BCValue(bcx_func, bcx_v_func, bcx_a_func)
    bcy = BCValue(bcy_func, bcy_v_func, bcy_a_func)
    bc.insert(n_beamAB_1, BC.Motion, [bc_clamp, bcx, bcy, bc_clamp, 
                                      bc_clamp, bc_clamp])
    \end{lstlisting}\end{tcolorbox}
    \item The similar structure is followed to define the mandrel as a stocky beam, but with fixed nodes (all $6$ DoF clamped) as:
    \pythonstyle
    \begin{tcolorbox}\begin{lstlisting}
    bc.insert(start_b2, BC.Motion, [bc_clamp, bc_clamp, bc_clamp, 
                                    bc_clamp, bc_clamp, bc_clamp])
    bc.insert(n_beamAB_2, BC.Motion, [bc_clamp, bc_clamp, bc_clamp, 
                                      bc_clamp, bc_clamp, bc_clamp])
    \end{lstlisting}\end{tcolorbox}
    \item The second yarn is defined in the similar structure as \pythoninline{beamAB}, but the circular motion is now in the opposite direction.
    \item Finally, the scaling and penalty parameters for the augmented Lagrangian are defined and the contact pairs are defined according to the convention that the first beam element passed to the contact element is the slave beam. Therefore, the yarn is defined as the slave and the mandrel as the master. Firstly, the contact pairs between the first yarn \pythoninline{nele_AB} and mandrel \pythoninline{nele_BC} are defined as:
    \pythonstyle
    \begin{tcolorbox}\begin{lstlisting}
    for i in range(nele_AB):
        for j in range(nele_BC):
            ps.create_element("Beam_Contact", elabel,
                              [i, i + 1, j + nele_AB + 1, j +
                              nele_BC + 2], contact_prop)
            elabel += 1
    \end{lstlisting}\end{tcolorbox}
    \item In a similar manner, the contact pairs are further defined between the mandrel and second yarn, with mandrel defined as master and the yarn as slave.
    \item The specified choice of the analysis scheme (\pythoninline{NonSmoothMortarStatic} is further defined and the mechanical analysis computes the solution using the \pythoninline{JSON} file.
    \pythonstyle
    \begin{tcolorbox}\begin{lstlisting}
    analysis.set_analysis_type(AnalysisType.NonSmoothMortarStatic)
    analysis.compute_solution(json_file)
    \end{lstlisting}\end{tcolorbox}    
\end{enumerate}

\subsubsection{JSON file}
In the \pythoninline{JSON} files, each of the components have at least two fields: \pythoninline{scheme} or \pythoninline{name} and \pythoninline{parameters}. The field \pythoninline{scheme} is used to identify numerical schemes and name is used to identify a component that is not a numerical scheme. The \pythoninline{field} parameter is used to specify any parameter needed to configure the specific scheme or name identifying the component to be created. You can observe that as scheme the \pythoninline{Newton_NL_Solver} is provided, which identifies as the Newton Nonlinear solver. As parameters, you have some properties which correspond to the nonlinear solver but also a \pythoninline{linear_solver} that in this case is set to \pythoninline{CSR_Pardiso_Solver}. This means that the direct intel MKL Pardiso solver is going to be used, assuming that the matrices are provided in CSR (Compressed Sparse Row) format. In addition, a component \pythoninline{Viewer} is defined and identified to name: \pythoninline{HDF5_Viewer}.

\subsection{Installation and running}
A Linux environment is preferred for Odin, although it can be used with a WSL ubuntu (or any virtual environment i.e. Oracle VirtualBox) on Windows. Additionally, \pythoninline{Visual Studio Code} is encouraged as the development environment, although any python notebooks (Jupyter Notebooks etc.) can be used. Odin has the option to automatically fetch and build the necessary third party libraries. Therefore, no additional installation procedures are needed. For the installation of Odin, please refer to the code documentation: \url{https://obruls.gitlabpages.uliege.be/doc4odin/firstSteps/index.html}.\\

Once Odin has been successfully installed along with the open source graphical software \pythoninline{Blender} for rendering, the python script example file can be run. It is important to note the \pythoninline{python} version being used to run the file, as it should be compatible with the \pythoninline{blender} version for rendering the simulation. The post-processing file reads an \pythoninline{hdf5} file format which is generated from the simulation. This \pythoninline{.h5} file is further imported into a python script written in \pythoninline{Blender} for the visualization of the simulation using \pythoninline{Pyblender}. 