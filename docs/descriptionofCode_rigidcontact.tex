\subsection{Description of code: beam-to-rigid body contact}
Odin multibody dynamics research code \cite{odin2022} is written in C++ programming language. The philosophy of Odin relies on the use of nonlinear finite elements to represent the components of a multibody system. This is further accompanied by kinematic or contact constraints. The equations of motion are automatically assembled in Odin based on the specified choice of the mechanical analysis and further solved using the specified time integration scheme.\\

\subsubsection{Python code file} 
\begin{enumerate}
    \item The necessary modules are imported and Odin is initialized using \pythoninline{initialize_odin()}.
    \item The variables for radius of spherical collision geometries \pythoninline{r_sphere}, radius of beam \pythoninline{r}, length \pythoninline{l} and mandrel dimensions are defined.
    \item The \pythoninline{JSON} file which configures the methods to be used (time integrator, nonlinear solvers etc.) is loaded using \pythoninline{json_file = json.load(open())}.
    \item A physical system is created which is the master container for all the components of the system, using \pythoninline{ps = create_physical_system()}. Additionally, the mechanical analysis and nodal frames are defined in the physical system. As a nodal frame is defined by translations and rotations, a \pythoninline{Vec3} (vector of 3 components) can be used for defining translations along with quaternions for rotations.
    \item The collision properties (\pythoninline{collision_props}) for rigid body contacts are defined using \pythoninline{RigidBodyContactPropsSet()}. The penalty and scaling parameters are then assigned to the property set along with the coefficients of restitution for normal \pythoninline{e_n}, tangential directions \pythoninline{e_t}, and friction \pythoninline{mu}.
    \item Using the defined  \pythoninline{collision_props}, a collision group for point-to-plane contacts is created as:
        \pythonstyle
        \begin{tcolorbox}\begin{lstlisting}
            cs = ps.create_collision_group("Rigid_Body_Contact", 
            collision_props)
        \end{lstlisting}\end{tcolorbox}
    \item Further, the collision system (\pythoninline{cs}) can be used to define the collision shapes for the mandrel and spherical geometries as:
        \pythonstyle
        \begin{tcolorbox}\begin{lstlisting}
            ground_shape = cs.create_Capsule_Shape(boxd, l)
            sphere_shape = cs.create_Sphere_Shape(r_sphere)
        \end{lstlisting}\end{tcolorbox}
    where, the \pythoninline{ground_shape} refers to the mandrel geometry defined as a capsule (cylinder) and the \pythoninline{sphere_shape} defines the proxy collision geometries to be attached to the nodes of the beam. For the collision detection, \pythoninline{Bullet} engine is used.
    \item The material properties and sectional stiffness coefficients of the beam are defined using the variable \pythoninline{params}. Using \pythoninline{beam_props}, the choice of using a tangent operator and geometric stiffness is defined with \pythoninline{True} or \pythoninline{False} as \pythoninline{beam_props.with_tg_operator = False}
    \item The normal (\pythoninline{n_0}) and binormal (\pythoninline{b_0}) vectors of the orthonormal triad are defined.
    \item The beam is created by looping through the number of elements (\pythoninline{num_elems}) and then defining spherical collision elements attached to the node as:
    \pythonstyle
    \begin{tcolorbox}\begin{lstlisting}
        for i in range(num_elems):
            dr = start_point + (i + 1) * d / num_elems
            start_node_label += 1
            ndl.insert_nodal_frame(start_node_label, dr[0], dr[1], 
            dr[2], q)
            sphere_objs.append(
                cs.create_collision_object(sphere_shape, pos, 
                orientation))
            ps.create_element("Beam", elabel, [start_node_label - 1,
            start_node_label], beam_props, [10], 
            [sphere_objs[-2], sphere_objs[-1]])
            elabel += 1
    \end{lstlisting}\end{tcolorbox}    
   where, using the relative distance (\pythoninline{dr}) between the elements, a nodal frame is inserted at \pythoninline{start_node_label}. Further, a spherical geometry is appended into the list \pythoninline{sphere_objs} which is further used during \pythoninline{ps.create_element} for \pythoninline{"Beam"}. The element label \pythoninline{elabel} is incremented using elabel += 1 and \pythoninline{[10]} is a classifier ID assigned to each element for reference.
    \item For the mandrel, \pythoninline{RigidBodyProps()} are assigned and a rigid body element is created using a classifier ID of \pythoninline{[298]} as \pythoninline{ps.create_element("Rigid_Body", elabel, [start_node_label], prop_ground, [298], [ground])}. 
    \item Essential boundary conditions for circular motion are now defined using piecewise linear functions \pythoninline{PWLFunction()} for position, velocity and acceleration levels as:
        \pythonstyle
        \begin{tcolorbox}\begin{lstlisting}
            bc_clamp = BCValue(0, 0, 0)
            bc = analysis.get_essential_bcs()
            bc.insert(0, BC.Motion, [bc_clamp, bcy, bcx, 
                                     bc_clamp, bc_clamp, bc_clamp])
            bc.insert(n_beamAB_1, BC.Vec_XYZ, [bc_clamp, bc_clamp, 
                                               bc_clamp])
            bc.insert(fix_ground, BC.Motion,
                                    [bc_clamp, bc_clamp, bc_clamp, 
                                     bc_clamp, bc_clamp, bc_clamp])
        \end{lstlisting}\end{tcolorbox}
    The prescribed circular motion is defined for the start node of the beam (\pythoninline{nlabel = 0}). Further, using \pythoninline{BC.Vec_XYZ} which sets the rotation DoF's free, the end node (\pythoninline{n_beamAB_1}) is restrained for translation (3 DoF's only). The mandrel node (\pythoninline{fix_ground}) is fixed (all 6 DoF clamped).        
    \item The specified choice of the time integration scheme is defined as (\pythoninline{NonSmoothDynamic} and the mechanical analysis computes the solution using the \pythoninline{JSON} file. The specific parameters for the time integration are defined in the \pythoninline{JSON} file.
        \pythonstyle
        \begin{tcolorbox}\begin{lstlisting}
            analysis.set_analysis_type(
                            AnalysisType.NonSmoothDynamic)
            analysis.compute_solution(json_file)
        \end{lstlisting}\end{tcolorbox}    
\end{enumerate}

\subsubsection{JSON file}
The time integration scheme in the \pythoninline{JSON} file is specified as \pythoninline{NSGA_GS} which identifies as the nonsmooth generalized-$\alpha$ time integration scheme along with the Gau{\ss}-Seidel solver (So-bogus). The maximum iterations are set to 50. The nonlinear solver to be used within the generalized alpha time integrator is set to \pythoninline{Newton_NL_Solver}, which identifies as the Newton Nonlinear solver. As parameters, you have some properties which correspond to the nonlinear solver but also a \pythoninline{linear_solver} that in this case is set to \pythoninline{CSR_Pardiso_Solver}. This means that the direct intel MKL Pardiso solver is going to be used, assuming that the matrices are provided in CSR (Compressed Sparse Row) format. In addition, a component \pythoninline{Viewer} is defined and identified to name: \pythoninline{HDF5_Viewer}.

\subsection{Installation and running}
For the installation of Odin, please refer to the code documentation: \url{https://obruls.gitlabpages.uliege.be/doc4odin/firstSteps/index.html}. Odin has the ability to fetch and install the third-party libraries automatically.\\

For the installation of So-bogus library, please refer to the documentation: \url{http://gdaviet.fr/doc/bogus/master/doxygen/}. In some cases, the installation of So-bogus might have to be performed manually referring to the documentation. After the successful installation of So-bogus, from the \pythoninline{build} directory in Odin, turn \pythoninline{ON} the CMake option of \pythoninline{WITH_SO_BOGUS} before compiling Odin. Once Odin is compiled and installed with the CMake option \pythoninline{WITH_SO_BOGUS} to \pythoninline{ON}, the python example script can be run. The post-processing file reads an \pythoninline{hdf5} file format which is generated from the simulation. This \pythoninline{.h5} file is further imported into a python script written in \pythoninline{Blender} for the visualization of the simulation using \pythoninline{Pyblender}.

